% Define a new command to make the command descriptions uniform
\newcommand{\glcommand}[6]{

\subsection{#1}
\leftskip2em #2\\
\\
\leftskip2em \textbf{Syntax:}\\
\\
\leftskip2em \texttt{#3}\\
\\
\leftskip2em \textbf{Parameters:}
\begin{itemize}
\leftskip2em #4
\end{itemize}
\leftskip2em \textbf{Example:}\\
\\
\leftskip2em \texttt{#5}\\
\\
\leftskip2em #6\\
}


% template for a commands copy, paste, use editor's Uncomment command to remove % characters, then replace placeholders
% if you get a missing $ error message, you likely forgot to use \_ instead of _...
%\glcommand{COMMAND_NAME}
%{
% DESCRIPTION
% }{
% SYNTAX
% }{
% \item LATEX \item LIST WITH PARAMETERS
% }{
% EXAMPLE
% }{
% COMMENTS FOR EXAMPLE
% }


\chapter{The commands}
The following sections describe the commands of the gsas\_language. They are arranged by topic, an alphabetic list is also provided.

\section{Complete list of commands}

\begin{verbatim}
gsas_add_atom
gsas_add_diffuse_scattering
gsas_add_histogram
gsas_add_phase
gsas_calc_bond_length
gsas_change_atom
gsas_change_background
gsas_change_DIFC
gsas_change_Fobs_extraction_flag
gsas_change_lattice
gsas_change_phase_flag
gsas_change_phase_scale
gsas_change_sigma1
gsas_change_sigmas
gsas_constrain_atom
gsas_constrain_phase
gsas_constrain_sigma1
gsas_convert_atom_thermal
gsas_copy_expfile
gsas_delete_atom_constraint
gsas_delete_sigma1_constraint
gsas_done
gsas_exclude_region
gsas_fourier_maps
gsas_initialize
gsas_plot
gsas_plot_bond
gsas_plot_overview
gsas_prepare_mem
gsas_read_phase
gsas_refine
gsas_replace_histogram
gsas_simulate_histogram
gsas_single_peak_fits
gsas_single_peak_fits_make_list
gsas_vary_absorption
gsas_vary_atom
gsas_vary_background
gsas_vary_DIFC
gsas_vary_diffuse_scattering
gsas_vary_histogram_scale
gsas_vary_lattice
gsas_vary_phase
gsas_vary_sigma1
gsas_vary_UVW
gsas_waterfall_add
gsas_waterfall_plot
\end{verbatim}

% Added by Nina: (Maybe you want to double check them)
%
% gsas_constrain_phase
% gsas_constrain_atom (still need to add something about multi)
% gsas_vary_phase
% gsas_convert_atom_thermal
%%
% Commands that need documenting, delete from here when done...
%
% gsas_add_diffuse_scattering
% gsas_calc_bond_length
% gsas_change_DIFC
% gsas_change_Fobs_extraction_flag
% gsas_change_phase_flag
% gsas_change_phase_scale
% gsas_constrain_sigma1
% gsas_copy_expfile
% gsas_delete_atom_constraint
% gsas_delete_sigma1_constraint
% gsas_exclude_region
% gsas_plot
% gsas_plot_bond
% gsas_plot_overview
% gsas_prepare_mem
% gsas_replace_histogram
% gsas_simulate_histogram
% gsas_single_peak_fits
% gsas_single_peak_fits_make_list
% gsas_vary_absorption
% gsas_vary_atom
% gsas_vary_background
% gsas_vary_DIFC
% gsas_vary_diffuse_scattering
% gsas_vary_histogram_scale
% gsas_vary_lattice
% gsas_vary_UVW
% gsas_waterfall_add
% gsas_waterfall_plot

\section{Beginning and end of refinement}

%\glcommand{1}{2}{3}{\item 4}{5}{6}

\glcommand{gsas\_initialize}
{
This command initializes a new refinement by generating a new GSAS EXP file. It deletes temporary files of possible previous refinements and initializes the documentation file of this refinement.
}{
gsas\_initialize <EXP filename> <title>
}{
\item \texttt{<EXP filename>}: The filename for the GSAS EXP file and assorted files produced by POWPREF, GENLES, etc. Filename must not have an extension. Filename will be automatically converted to capital letters since GSAS expects capital letters on UNIX platforms.
\item \texttt{<title>}: An arbitrary title to be displayed in the POWPLOT plots etc.
}{
gsas\_initialize SIO2\_FURNACE\_\$1C "SiO2, furnace, \$1C"
}{
When a script with this line is called with a parameter 100, e.g. the GSAS data file for a run at  100C, 
named 100C.gda, the \$1 occurences will be replaced with "100" by bash, generating a EXP filename 
SIO2\_FURNACE\_100C.EXP with a run title label "SiO2, furnace, 100C".
}

\glcommand{gsas\_done}
{
This command finishes a refinement. Plots of $\chi^2$, normalized shifts, time per iteration and number of variables are generated using gnuplot and included in the final Acrobat PDF file describing the refinement. Several checks (negative thermal motion, parameters that are zero within their error bars) are performed and added to the refinement report. Temporary files are deleted. If a refinement is aborted and the user wishes to see the overview PDF file, gsas\_done can be issued from the command line.
}{
gsas\_done
}{
\item None.
}{
gsas\_done
}{
This would finish a refinement.
}

\section{Adding histograms and phases}

\glcommand{gsas\_add\_atom}
{
Adds an atom to an existing phase of the refinement. The phase can be either added using the gsas\_read\_phase or gsas\_add\_phase commands.
}{
gsas\_add\_atom <phase number> "<GSAS atom parameter sequence>"
}{
\item \texttt{<phase number>}: The number of the phase in the main EXP file to which this atom is to be added.
\item GSAS atom parameter sequence: Space-separated sequence of element symbol, x, y, z coordinates, site occupation factor, atom label (or forward slash for default), thermal motion flag and value of thermal motion parameter(s). This is the same sequence a user would provide following a "i n" command in the atoms menu in EXPEDT.
}{
gsas\_add\_atom 1 "NI 0 0 0 1 / i 0.004"
}{
This would add a nickel atom to phase 1 on position x=0, y=0, z=0 that is fully occuped (FRAC=1) with the default name that GSAS assigns (the element symbol plus the number of the atom in the sequence). The atom would have isotropic thermal motion with a starting value of $U_{iso}=0.004$.
}

\glcommand{gsas\_add\_histogram}
{
Adds a phase to the refinement by adding the phase information "manually" rather than loading all information from a file. Similar to the addition in GSAS, no atom information is provided in this step (see gsas\_add\_atom). The existence of the space group and the lattice parameters in the main EXP file is verified after adding this information. Existence of the data and the instrument parameter file are checked before the insertion of the histogram data is attempted. Existence of the histogram data filename in the EXP file is checked after the insertion. This command may be used multiple times if multiple histograms are required for a refinement.
}{
gsas\_add\_histogram <data\_file> <par\_file> <bank> <min\_d> <max\_d>
}{
\item \texttt{<datafile>}: Filename of the data file with the histogram data to be loaded (GSAS data file format).
\item \texttt{<par\_file>}: Filename of the instrument parameter file required to interpret the histogram data.
\item \texttt{<bank>}: Bank number of the histogram to be read.
\item \texttt{<min\_d>, <max\_d>}: Minimum and maximum d-spacing to be used for this histogram.
}{
gsas\_add\_histogram nickel.raw inst\_tof.prm 2 0.25 3
}{
This would add histogram number 2 from the GSAS data file nickel.raw, using the GSAS instrument parameter file inst\_tof.prm. The minimum d-spacing for this histogram would be set to 0.25\AA\ and the upper d-spacing limit would be 3\AA.
}


\glcommand{gsas\_add\_phase}
{
Adds a phase to the refinement by adding the phase information "manually" rather than loading all information from a file. Similar to the addition in GSAS, no atom information is provided in this step (see gsas\_add\_atom). The existence of the space group and the lattice parameters in the main EXP file is verified after adding this information.
}{
gsas\_add\_phase "<phase name>" "<space group>" "lattice parameters"
}{
\item \texttt{phase name}: The identifier for the phase in GSAS.
\item \texttt{space group}: The space group of the crystal structure.
\item \texttt{lattice parameters}: The lattice parameters for the crystal structure.
}{
gsas\_add\_phase "Nickel" "f m -3 m" "3.5234"
}{
This will add a phase with the label \texttt{Nickel} that has the cubic space group $Fm\bar{3}m$ and a lattice parameter of 3.5234\AA.
}

\glcommand{gsas\_read\_phase}
{
Adds a phase to the refinement by loading a crystal structure from a file. The file with the crystal structure information has to be in GSAS format, CIF files are currently not supported. The existence of the phase file is checked before loading is attempted. The number of the phase during the refinement is determined by the sequence by which the phases are added to the refinement.
}{
gsas\_read\_phase <filename> <phase number>
}{
\item \texttt{filename}: The name of the file (with extension, e.g. \texttt{.exp}) from which the crystal structure information is to be read.
\item \texttt{phase number}: The number of the phase in the GSAS EXP file, typically 1.
}{
gsas\_read\_phase "Boron\_beta\_68106.exp" 1
}{
This would read phase information (lattice parameters, space group, atom parameters such as positions and thermal motion parameters) from a file name \texttt{Boron\_beta\_68106.exp} (case sensitive!). In that file, phase number 1 would be read.
}

\section{Crystal structure refinement}

\glcommand{gsas\_change\_atom}
{
This command is used to change values of atom parameters, such as position or thermal motion, but also the atom type to change the element/isotope for a given atom or atom sequence. Using the EXPEDT convention to address several atoms at once, e.g. in the sequence of atoms within a phase or by their element type, multiple atoms can be affected at once.
}{
gsas\_change\_atom <phase> <atom range> <parameter name> <value>
}{
\item \texttt{phase}: The number of the phase to which the atoms belong that need to be changed.
\item \texttt{atom range}: An atom range, e.g. the number of a single atom, a sequence in the GSAS atom list of the phase, or a specific type (element or isotope). This parameter can be anything EXPEDT understands.
\item \texttt{parameter name}: The parameter name, one of \texttt{X, Y, Z, UISO, FRAC, U11, U22, U33, U12, U13, U23, TYPE, NAME}.
\item \texttt{value}: The new value for the parameter of the atoms addressed with the previous parameters. 
}{
gsas\_change\_atom 1 NI UISO 0.004
}{
This would change the isotropic thermal motion parameter \texttt{UISO} of all atoms of the type \texttt{NI} in phase 1 to 0.004\AA$^2$.
}

\glcommand{gsas\_change\_background}
{
This command changes the type of the background function used for a given histogram and the number of parameters for that function.
}{
gsas\_change\_background <histogram> <background function> <number of parameters>
}{
\item \texttt{histogram}: The number of the histogram for which the background will be changed.
\item \texttt{background function}: The number of the function to be used to describe the background in this histogram. Currently, valid numbers are 1, 2, 4, 5, 6, 7, and 8 (see page 129 in the 2004 or 'modern' GSAS manual).
\item \texttt{number of parameters}: The number of parameters or coefficients to be used for the selected background function. GSAS supports up to 36 parameters, typically much less are sufficient.
}{
gsas\_change\_background 2 1 16
}{
Changes the background of histogram 2 to GSAS background function number 1 with 16 parameters.
}

\glcommand{gsas\_change\_lattice}
{
Changes the lattice parameters of a given phase. This might be necessary for instance when the parameters from the phase file read with gsas\_read\_phase do not match the experimental data, e.g. due to differences in the chemistry.
}{
gsas\_change\_lattice <phase> <lattice parameters>
}{
\item \texttt{phase}: The number of the phase for which the lattice parameters are changed.
\item \texttt{lattice parameters}: The new lattice parameters. The number of parameters depends on the space-group of the phase, e.g. one for cubic, two for hexagonal, or six for triclinic crystal structures. The parameters are input as in GSAS, e.g. first $a$, then $c$ for hexagonal, or in the sequence $a, b, c, \alpha, \beta, \gamma$ for triclinic phases. The parameters can be either given as individual numbers, or as a single string encapsulated in double-quotes.
}{
gsas\_change\_lattice 1 3.14 11.05 OR
gsas\_change\_lattice 1 "3.14 11.05"
}{
Sets the lattice parameters of phase 1 to 3.14A and 11.05A. Since there are two parameters given, the phase is either hexagonal, tetragonal, or rhombohedral.
}

\glcommand{gsas\_convert\_atom\_thermal}
{
Converts thermal motion parameters of atoms to isotropic (UISO) or anisotropic (U11, U22, U33, etc).
}{
gsas\_convert\_atom\_thermal <phase\#> <atom range> <flag a/i>
}{
\item \texttt{phase\#}: Number of phase for which parameters should be changed.
\item \texttt{atom range}: Atom(s) for which thermal factors should be changed.  This is an atom range, e.g. the number of a single atom, a sequence in the GSAS atom list of the phase, or a specific type (element or isotope). This parameter can be anything EXPEDT understands. \texttt{<t>}, \texttt{<s>} or \texttt{<s1:s2>}
\texttt{flag A/I}:  Flag for thermal factors: \texttt{i} for isotropic or \texttt{a} for anisotropic. \\
}{
gsas\_convert\_atom\_thermal 1 1:4 a
}{
This would convert the thermal motion parameters to anisotropic for atoms 1, 2, 3 and 4 of phase \#1.
}

\glcommand{gsas\_constrain\_atom}
{
This command is used to set a given atom parameter (e.g. occupancy, position, thermal motion) for different atoms in a phase to be constrained together during refinements.  In most cases, this would set the different atom parameters to vary together (for example, if you wanted the thermal motion of atoms \# 2 and \#3 to be constrained together).  Note that if you want to atom parameters to be equal, they must start out equal before they are constrained together, otherwise they will vary proportional to one another.  In the case of constraining the occupancy, the values will be constrained such that they alway add up to the sum of the their starting occupancies (in many cases, 1). 
 }{
gsas\_constrain\_atom <phase\#> <atom prm> <atom\#s>
 }{
 \item \texttt{phase\#}:  Phase number for which atom parameters should be constrained. 
\item \texttt{atom prm}: Atom parameters which should be constrained, one of \texttt{X, Y, Z, UISO, FRAC, U11, U22, U33, U12, U13, U23}.
\item \texttt{atom\#s}:  Atom numbers of the atom set that should be constrained, listed separately.
 }{
gsas\_constrain\_atom 1 U11 2 3 
 }{
Sets the thermal motion parameter U11 for atoms 2 and 3 in phase 1 to be constrained together.
 }

\glcommand{gsas\_contrain\_phase}
{
Constrains a phase to have a constant weight fraction in all histograms.  In most cases, this should be set before varying a phase during refinements. 
 }{
gsas\_contrain\_phase <phase\#>
 }{
 \item \texttt{phase\#}  Phase number for which weight fraction should be constrained.
 }{
gsas\_contrain\_phase 2
 }{
Sets the weight fraction of phase 2 to be equal in all histograms. 
 }

\glcommand{gsas\_vary\_phase}
{
Sets refinement flag for weight fraction of specified phase.  
}{
gsas\_vary\_phase <phase\#> <flag>
}{
\item \texttt{phase\#}:  Number of phase for which weight fraction should be varied or fixed. 
\item \texttt{flag}: Flag for varying phase: \texttt{<y>} to vary or \texttt{<n>} to fix; if no flag then assume \texttt{<y>}.  
}{
gsas\_vary\_phase 2 y
}{
Sets refinement flag for phase 2 to ``y'' (vary it).
}

\glcommand{gsas\_refine}
{
Performs the actual refinement by calling POWPREF and GENLES for a given number of refinement cycles. Each call to this command adds a section to the refinement protocol generated at the very end of the refinement.
}{
gsas\_refine <cycles> [noplot] [nopowpref]
}{
\item \texttt{cycles}: The number of refinement cycles in GENLES to be performed.
\item \texttt{noplot}: If the word "noplot" is added after the number of cycles, the time-consuming generation of plots using POWPLOT is skipped.
\item \texttt{noplot}: If the word "nopowpref" is added after the number of cycles, running of \texttt{powpref} will be skipped. This is useful during Le Bail fits during which re-setting the peak intensities by \texttt{powpref} may lead to divergence.
}{
gsas\_refine 2 noplot
}{
Performs a least-squares cycle by calling POWPREF, then changing the number of refinement cycles to 2 and calling GENLES. Plots of the fit will not be generated for the overview file.
}

\glcommand{gsas\_vary\_DIFC}
{
Changes the refinement flags for the diffractometer constants (DIFCs) for a given histogram. This is needed to e.g. calibrate an instrument or sample environment or to accomodate sample misalignment.
}{
gsas\_vary\_DIFC <histogram> <code>
}{
\item \texttt{histogram}: The number of the histogram for which the new refinement flags are given.
\item \texttt{code}: The code(s) of the diffractometer constants that should be varied. These depend on the instrument type, e.g. for neutron time-of-flight they are \texttt{C} for \texttt{DIFC}, \texttt{A} for \texttt{DIFA}, and \texttt{Z} for \texttt{ZERO}. The codes for other instrument types can be queried with \texttt{expedt}, options \texttt{k l o c v}. Combinations are possible, e.g. \texttt{CA} would vary \texttt{DIFC} and \texttt{DIFA} in the above example. To fix diffractometer constants use \texttt{" "} (double quotes with space inbetween) as code.
}{
gsas\_vary\_DIFC 1 C
\newline
gsas\_vary\_DIFC 2 " "
}{
Turns on variation of \texttt{DIFC} for histogram number 1 and fixes all diffractometer constants for histogram 2.
}

\section{Neutron time-of-flight specific}

\glcommand{gsas\_change\_sigmas}
{
Changes the values for the three peak width parameters of neutron time-of-flight profile function number 1. This function can be used to reset the values if needed or initialize to a specific value in case constraints between peak width parameters are introduced that require a certain ratio, e.g. that they are the same, between peak width parameters of different histograms or phases.
}{
gsas\_change\_sigmas <histogram> <phase> <sigma0> <sigma1> <sigma2>
}{
\item \texttt{histogram}: The number of the histogram for which the value of the $\sigma$ parameters should be changed.
\item \texttt{phase}: The number of the phase for which the value of the $\sigma$ parameters should be changed.
\item \texttt{sigma0,sigma1,sigma2}: The value for each of the three parameters.
}{
gsas\_change\_sigmas 2 1 0 60 0
}{
Changes the peak width parameters $\sigma_0$, $\sigma_1$, and $\sigma_2$ for histogram 2, phase number 1, to 0, 60, and 0, respectively.
}

\glcommand{gsas\_vary\_sigma1}
{
Variation of the $\sigma_1$ peak-width parameter for neutron time-of-flight data using profile function number 1, the default for many neutron time-of-flight diffractometers.
}{
gsas\_vary\_sigma1 <histogram> <phase> <flag> <damping>
}{
\item \texttt{histogram}: The number of the histogram for which the $\sigma_1$ parameter should be varied.
\item \texttt{phase}: The number of the phase for which the $\sigma_1$ parameter should be varied.
\item \texttt{flag}: The refinement flag, either \texttt{y} for varyiable or \texttt{n} for fixed.
\item \texttt{damping}: The damping flag, a number between 0 (no damping) and 9 (maximal damping).
}{
gsas\_vary\_sigma1 2 1 y 4
}{
Varies (\texttt{y}) the $\sigma_1$ parameter for histogram 2, phase 1 and damps the variation with a damping factor of 4.
}

\section{Generating extra infomation}

\glcommand{gsas\_fourier\_maps}
{
Generates all Fourier maps defined in GSAS: Calculated (FCLC), observed (experimental, FOBS) and difference Fourier map (DELF) as well as a 2FDF map (the Fourier map $2F_o-F_c$) . After generation, the Fourier maps are output in \texttt{.grd} format, which can be imported for instance into Koichi Momma's software VESTA for viewing Fourier maps in 3D together with the crystal structure for comparison and e.g. identification of potentially missing atoms. You may download VESTA from

\url{http://www.geocities.jp/kmo_mma/crystal/en/vesta.html}

Open the grd file in VESTA. To modify the contour levels in VESTA, go to Objects -- Properties -– Isosurfaces, select an isosurface in the list (roughly center of the dialog box, next to new, delete and clear buttons) and modify the level, color etc. above the list. In order to also plot the atoms, go to Edit Data -– Structure Parameters and click on import (bottom right) and read the EXP file with the current crystal structure. If you save a VESTA file with a .grd and your current imported EXP file, you may just re-open the VESTA file to obtain updated structure plots after refinement and/or generation of Fourier plots have run again.
}{
gsas\_fourier\_maps <section>
}{
\item \texttt{section}: The type of section for the Fourier map, x, y, or z. The default (if this parameter is not given) is z.
}{
gsas\_fourier\_maps
}{
This will simply generate all Fourier maps in \texttt{.grd} format.
}

\section{Plotting of results}


