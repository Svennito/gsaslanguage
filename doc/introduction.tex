\chapter{Introduction}

Automation of data analysis of diffraction data is arguably one of the biggest challenges since the advent of high throughput facilities such as modern synchrotron or neutron facilities. Besides shear sample throughput, nowadays in many cases parametric studies are undertaken during which hundreds or thousands of runs at different pressures, temperatures, stress or other external parameters are acquired. Similarly, for studies of kinetics, data are taken at fixed short time intervals for prolonged periods, generating large numbers of runs that need to be analyzed. While Rietveld analysis \cite{rietveld1969profile,young1996rietveld} has become the standard way to extract crystallographic information from such data, automation of such data analysis is by no means a trivial task and frequently the obstacle for inexperienced or even experienced users of the Rietveld technique towards publication. While it was suggested to incorporate the parameters describing a series of datasets during a parametric study, e.g. the coefficient of thermal expansion or kinetic parameters of phase transformations, into the refinement of the whole dataset (\cite{stinton2007parametric}, \cite{halasz2010parametric}), this technique can not be generalized and is not available in mainstream Rietveld packages. On the other hand, the role of beam line personnel at large scale facilities also includes, or should include, assistance of the users with data analysis. In view of the large numbers of user visits at many facilities nowadays, this task demands efficient knowledge transfer on e.g. the refinement strategy for the particular type of data from a particular instrument from experts to users with experience ranging from non-existent to very advanced, but for other instruments. The analysis of powder diffraction data using a complex model with tens or in some cases even hundreds of parameters makes this analysis technique somewhat and outlier from other materials characterization techniques with much smaller numbers of parameters. Therefore, for beginners it is often cumbersome to understand the complex interplay of parameters describing a diffraction dataset. While the general rule that the next parameter to be varied on should be the one that is deemed to have the largest contribution to difference curve holds for specific instruments and techniques, the sequence and type of parameters that need to be varied can be considered specific to the instrument. Guidance on the topic of refinement strategies exists for users of the Rietveld method (e.g. \cite{mccusker1999rietveld}), even for specialized topics like magnetic structure analysis using GSAS \cite{cui2006magnetic}, but highly specialized instruments or sample environments will always require special details that need to be communicated to every user. The learning curve for the particular Rietveld codes can also be fairly steep for the beginner, and while this issue can be partially circumvented by hiding the input- and output-specific problems behind a graphical user interfaces (e.g. \cite{toby2001expgui}), such interfaces do not assist in automation of the analysis. Some refinement techniques, such as the determination of site occupation factors as described by \cite{heuer2001determination}, require repetitive refinements with slight variations of e.g. the d-spacing range used in the refinement and could possibly be more wide-spread used if automation would be more readily available. Similarly, occasionally it is required to try several proposed structures and compare their match with a given dataset, providing another application of automation to ensure that the identical refinement strategy was applied with each structure model. Frequently, documentation of the applied refinement strategy is not available and users may even find themselves unable to reproduce a particular successful refinement. On the other hand, black-box automation of refinement with minimal user interaction may cause problems even for experienced users when the hard-coded refinement strategy does not work. Guidance of the user from e.g. the correlation matrix or the evolution of the reduced $\chi^2$ or the parameter shift, that may indicate refinement problems, are typically not obvious for beginners but may contain very valuable indications as to what possible refinement problems might be. Finally, there is no standard way to communicate and exchange successful refinement strategies that would be useful to at least serve as a starting point for other users and problems or in order to remote advise to a user by an experienced user.

In this paper, we describe our approach to address these problems, allowing users to focus on the actual refinement rather than having to deal with the specific obstacles of adding histograms or phase information to the refinement.  To address the above issues, we developed a script language that allows to program a refinement of diffraction data using the GSAS \cite{larson2004gsas} package as the back-end. These scripts can be re-used with minimal modifications, either for analysis of similar datasets during a parametric study, or for similar experiments undertaken by other users. They generate an overview file for documentation of the refinement process as well as initial plots of graphs of any refined parameter. The scripts are using the bash shell, both to program commands controlling the analysis as well as programming the actual analysis, and are therefore available on Windows (using the cygwin package, available for free at www.cygwin.com), and for Mac or Linux without installing any software other than the collection of scripts and possibly Latex or gnuplot and of course GSAS. The bash environment in combination with Unix text processing tools allows validation of input (e.g. existence of input files such as instrument parameter files or data files), powerful on-the-fly extensions (no compilation necessary), as well as efficient extraction of results from various GSAS output files (list files, parameter-value-esd files etc.). If other script languages, e.g. Tcl/Tk are installed, crystallographic codes coded in these languages, can be easily included. Using a script language allows also to implement fairly specific checks and validations on common mistakes, such as refinement of the ZERO instrument parameter instead of DIFC in neutron time-of-flight refinements or variation of calibration parameters and lattice parameters, and to provide the user with feedback and warnings. The bash shell creates very little overhead to the execution of the GSAS commands, allowing for efficient analysis of large numbers of datasets. Conditional refinements, i.e. variation of certain phase specific parameters only above a certain threshold for the weight/volume fraction of the phase, can be implemented. 
